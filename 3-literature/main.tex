\chapter{Literature Review}\label{c:literature}

\section{Optrode system}

Optrodes have come to the forefront of the scientific community as advanced tools for enhancing our understanding of bioelectric signals. The recent advancements in this field pave the way for the possibility of an active noise-cancelling system tailored for optrode devices. A deeper look into the most recent research elucidates this potential.

Historically, electrophysiology methods were reliant on electrodes and wires for sensing and transmission of bioelectric signals. Such electronic methods inherently faced issues like limited bandwidth, design complexities in signal conditioning circuits, and constraints in the number of channels, especially for more invasive or miniature devices. Addressing these limitations, an innovative approach utilizing light for both sensing and transmission was introduced \cite{OptrodeTransducer}. This method leveraged the birefringence property of liquid crystals (LCs) operating on a microscale to convert bioelectric signals into the optical domain. This provided a significant leap in terms of linearity, relative responsivity, and bandwidth in capturing electrophysiological signals, with successful demonstrations in capturing signals from both cardiac and neural sources.

Delving deeper into the optical methodologies for bioelectric signal capture, there was a noteworthy proposition for an 'optrode' sensor array designed for biopotential measurements\cite{OptrodeArray}. The uniqueness of this system stemmed from its transduction mechanism, which utilized deformed helix ferroelectric liquid crystals. These crystals underwent realignment in response to potential fields of biological cells, impacting the optrode's light reflectance properties. A comprehensive computational model supported the utility of this optrode, especially in its capability for accurate transduction of neuronal spikes and capturing the propagation of neural impulses.

Beyond just signal capture, the accurate recording of electrically-excitable cells is crucial. This assists in decoding cellular networks and provides a better understanding of numerous electro-physiological dynamics. Here, the optrode technology presented itself as a beacon of hope, especially in rectifying impedance imbalances as recording sizes decrease \cite{ImpedanceOfOptrode}. A striking feature of this technology was its high impedance properties, ensuring minimal signal loss, even with diminished recording site sizes. This property is pivotal for achieving high spatial-resolution recordings, setting the stage for the development of passive optrode arrays with unprecedented accuracy.

Taking a different tangent, the exploration of SiO2 optrode arrays, for instance, has emphasized their role as penetrating waveguides beneficial for both optogenetic and infrared (IR) neural stimulation, with particular attention to light delivery and loss mechanisms \cite{MoreOptrode_1}.  Meanwhile, an integrated neural probe that combines optogenetic stimulation by microscale light-emitting diode (μLED) with concurrent neural activity recording has been brought to the fore. Notably, this device trumps its fiber-coupled laser-based optrode counterparts, particularly in closed-loop integration and in potential clinical settings \cite{MoreOptrode_2}. For more extensive mammalian behavioral studies, there's the electrically addressable optrode array, which capitalizes on a glass microneedle array synchronized with a microLED device to provide bi-level optogenetic neuron excitation in vivo \cite{MoreOptrode_3}. The world of optrode technology also witnessed the introduction of the three-dimensional (3D) drivable optrode array, utilizing laser diodes coupled waveguides, which has been lauded for its neural recording prowess, especially within intricate neural networks spanning multiple brain regions \cite{MoreOptrode_4}.

On the frontier of optogenetics in non-human primates (NHPs), the challenge has been creating multifunction optoelectronic probes that are both precise and trauma-minimizing. In response, the "coaxial optrode" emerged, tailored to carry out concurrent electrophysiology, light delivery, and fluorescence measurements in the NHP brain, setting a new standard in causing minimal cortical damage \cite{MoreOptrode_5}.

In conclusion, the rapid progression in the realm of optrodes offers a promising avenue in electrophysiology. Their capabilities in bioelectric signal conversion, coupled with high spatial-resolution recording potential, address many of the existing challenges in the field. Such insights and technological advancements are instrumental in envisioning an effective noise-cancelling system for optrode platforms, ensuring clear and interference-free data acquisition.

\subsection{Existing optrode system}
In this project, we are working on improve the noise performance of a existing optrode system \cite{OptrodeProceesings} (shown in Figure \ref{fig_OptrodeFigure}).  A superluminescent diode emits light to a circulator, which directs the light to a liquid crystal transducer.  The transducer receives nerve activity signal from the two electrodes.  This causes the change of liquid crystal angle, which then changes the reflection rate of the light.  Therefore, the nerve activity message is modulated to the light and the light is reflected to the circulator, which then directs the light to a photodiode light detector.  This detector converts the light signal to a current signal, and eventually the current signal is converted to digital data via an amplifier circuit and data acquisition equipment. 

\section{Active noise cancelling}

\subsection{Wiener filter}

\begin{figure}[htbp]
\centerline{\includegraphics[scale=0.5]{3-literature/Wiener_block.pdf}}
\caption{Wiener filter block diagram}
\label{fig_Wiener_block}
\end{figure}

To reduce the noise in the system, an active noise reduction algorithm is needed.  The wiener filter is one of the choices when dealing with two channel noise reduction.  A FIR wiener filter for discrete signals \cite{WienerPaper} is shown in Figure \ref{fig_Wiener_block}, s[n] is a noise free signal, and w[n] is s[n] plus some noise.  G(z) is the wiener filter with ai being the filter coefficients.   W[n] is inputted to the filter, and x[n] is the filter output.  e[n] is the error between x[n] and s[n].  From this diagram, it can be derived that the mean square error is as follows:
$E[e^2 [n]]=E[(x[n]-s[n])^2 ]$
After some derivatives and calculations, an equation came out:


\begin{gather} \label{eqn_WienerMatrix}
\underbrace{
    \begin{bmatrix}
    R_w[0] & R_w[1] & \dots & R_w[N] \\
    R_w[1] & R_w[0] & \dots & R_w[N-1] \\
    \vdots & \vdots & \ddots & \vdots \\
    R_w[N] & R_w[N-1] & \dots & R_w[0]
    \end{bmatrix}
}_{T}
\underbrace{
    \begin{bmatrix}
    a_0 \\
    a_1 \\
    \dots \\
    a_N
    \end{bmatrix}
}_{a}
=
\underbrace{
    \begin{bmatrix}
    R_{ws}[0] \\
    R_{ws}[1] \\
    \dots \\
    R_{ws}[N]
    \end{bmatrix}
}_{v}
\end{gather}

In eqn~\ref{eqn_WienerMatrix} the matrix T is the autocorrelation of w[n], and inside the matrix v is the cross correlation of w[n] and s[n], where both w[n] and s[n] is known.  Therefore, both matrix T and v are known, so matrix a can be calculated, which is the filter coefficients.
This method uses a noise free message signal s[n] and a noisy message signal w[n] as input, and the filter reduces the noise inside w[n].  This is different to how the noise reduction algorithm would work in this project.  From the optrode system, a noisy message signal and a pure noise signal can be extracted, while the pure noise signal is highly correlated to the noise inside the noise message signal.  To use the wiener filter, the pure noise signal can be the input w[n] and the noisy message signal can be the input s[n].  The output x[n] would be the correlated noise between the pure noise signal and the noisy message signal.  Then, subtract x[n] from s[n] to give the signal after noise reduction.
For two long discrete signals, say a 10s signal with a sampling rate of 100kHz, each signal has 1 million samples.  For matrix T, that means a 1 million times 1 million size, which takes a huge amount of computational power.  The plan is to run this algorithm on an FPGA, which cannot handle this task.  Also, it is desired to have real-time output with low latency.  The wiener filter is very hard to achieve both high noise reduction rate and low latency since the filter needs a fair number of samples to work.
In conclusion, the wiener filter is a good algorithm to start with.  however, it cannot suit all the requirements for this project, so modifications and other methods must be considered.



\subsection{ANC on audio systems}

In the pursuit of optimizing communication headsets for the transmission of bioelectric signals, an integrated ANC system has been introduced \cite{ANC_Headphone_1}. This system incorporates an adaptive feedback ANC filter to mitigate acoustic noise within the ear cups of the headsets and an adaptive noise-canceling filter to enhance near-end speech before transmission. The integration of these ANC techniques minimizes the need for excessive signal processing components, resulting in a streamlined and efficient solution for communication headsets.

Further contributing to the evolution of ANC, a study has explored the design and implementation of an adaptive feedback ANC system for headphone applications \cite{ANC_Headphone_2}. This research aimed to identify the optimal position for the error microphone within the ear-cup and employed music signals for adaptive system identification of the secondary path. The ANC headphone developed through this study excelled in noise cancellation, particularly in attenuating low-frequency harmonics.

In the context of in-car ANC systems, loudspeaker arrays are fundamental components \cite{ANC_Car}. Many in-car ANC systems utilize the vehicle's integrated loudspeakers to counteract noise from various sources, such as the engine. An analysis of the integrated loudspeakers' noise-cancelling capabilities has revealed promising results. The car's built-in loudspeakers could attenuate driving noise significantly, particularly for frequencies up to 500 Hz.

In summary, active noise cancelling methods have emerged as indispensable tools in the optimization of audio systems. These techniques have enabled the acquisition of accurate sound signals in the presence of environmental noise, enhancing the overall performance and reliability of sound-based applications.

\section{Digital signal processing on FPGA}

\subsection{Active Noise Cancellation for In-Ear Headphones Implemented on FPGA}

Reshma B proposed a hardware implementation of a headphone active noise cancellation algorithm on FPGA \cite{ANC_Headphone_3}.  The high-level design is basically a negative feedback control loop with inputs being reference microphone and error microphone, and output being sound data after noise reduction.  The feed back is connected to an adaptive filter that uses the FxLMS algorithm.  The coefficients are calculated by the previous coefficients, input data, and feedback error signal inside a weight update controller.  It keeps updating the coefficient until the error signal goes to zero.  Inside the adaptive filter, there are 24 filter taps.  There are results showing that after the 17h filter tap, the coefficients are very close to zero.  Also, a comparison was made showing that 24 filter taps and 60 filter taps achieve the exact same noise reduction ability, while 60 filter taps have 200\% more computational complexity.  Therefore, 24 is an appropriate number of filter taps that does not use too many resources but still maintains decent noise reduce ability.  A pipeline structure is also designed to reduce the hardware resources used.  As high frequency noise can be largely reduced by passive noise control, active noise cancellation only needs to work at lower frequencies, for instance, lower than 1kHz.  Modern FPGAs can run at around 100MHz frequency, which a lot faster than the required sound frequency.  In this case, 24 filter taps are being pipelined, which means for each sample period the filter is running 24 times to get 24 filter taps.  A single 24-Filter unit structure as well as the pipelined single-multiplier filter have been implemented to compare the resource utilization rate.  The result shows that the non-pipelined filter cost 77\% of slice LUTs and 425\% of bonded IOBs, and the pipelined filter cost 13\% of slice LUTs and 34\% of bonded IOBs.  It is not a 24 times difference since the pipelined filter is only part of the whole design.  However, it is still very significant as the non-pipelined design uses almost all the slice LUTs and 4 times the IO pins onboard, which means it is impossible to implement on the actual FPGA.
It is mentioned in this paper that the algorithm focuses on frequencies under 1kHz.  This is not only because passive noise control can handle the high frequency noise, but also active noise cancellation hardware now cannot deal with high frequency very well.  The algorithm and hardware need time to capture, process, and output.  Which means there must be a time delay between the input noise and the output cancelling noise.  This time delay may cause only a small phase shift at lower frequencies, but it would be very serious at higher frequencies, which leads to a failure to cancel or even add more noise.  For this project, the required working frequency range is about the same as audible sound, but does not have a strict time delay requirement, which means similar hardware and algorithm should satisfy the requirements of this project.  The filter length result suggests that there is a filter tap number after which the noise reduction ability will not change.  The optrode system data shows that the noisy signal and noise signal are only correlated for about 7 samples, so it is safe to assume that the noise reduction ability would reach its maximum at 7 filter taps.  This paper also discussed pipeline techniques, which could be very useful in this project.  In this project, more data bits may be used for higher accuracy, which means a lot more resources will be taken.  By using a pipeline structure, hardware cost is being converted to time cost.  Since the FPGA runs at a much higher frequency than the signal of interest, it is safe to run multiple times within a sample period.

\section{Photodiode}

In this project, we need to build a new PCB (a light receiver board) that converts analogue light signal into digital signal for the FPGA to process.  The input power to the light source in the existing optrode system just exceeds \qty{22.5}{mW} \cite{OptrodePower}.  We will be using the same photodiode from the existing optrode system from Flyin \cite{Flyin}, which has a responsivity of \qty{0.9}{A/W}.  we can then calculate the current output from the photodiode will be around $\qty{22.5}{mW}/\qty{0.9}{A/W}=\qty{25}{mA}$.  However, in the actual system, due to the light circuit configuration, the light loss in the light circuit, and the amount of light being modulated in the optrode, experiment shows only a few micro amps current coming out of the photodiode.  At this current level, we need to be very careful of the noise generated by the components around the photodiode, as they can easily making the signal to noise ratio bad.

  \subsection{Photodiode working in zero-mode: detecting light power change with DC rejection and AC amplification}

With the optrode system, the liquid crystal transducer modulates the nerve signal into the lights by changing the amount of light being reflected.  However, it only modulates a small portion of the light, which means the AC part of the light is far less than the DC part.  This may cause problems with the light receiver part of the system.
A common way to use photodiode as light detector is to use it in photoconductive mode.  In photoconductive mode, the photodiode is reverse biased, which means the anode has been given a voltage lower than the cathode.  A transimpedance amplifier is connected to the cathode of a photodiode to convert and amplify the output current to a voltage signal.  This is a good choice when it is needed for the output signal relevant to illuminance.  However, when only the AC part is needed, and the AC part is very small compared to the DC part, this DC coupled detect-amplify circuit can be saturated without amplifying the circuit to a satisfactory level.  This means more noise is added to the signal, harder for AC coupled amplify, and worse performance for the noise reduction algorithm.
In this paper, Yuan Wei and his team proposed a new mode to operate photodiodes \cite{zero-mode_detection}, which results in higher AC gain and lower noise.  This mode is called “zero mode”, and it forces the photodiode to operate at either zero voltage or zero current.  

\begin{figure}[htbp]
\centerline{\includegraphics[width=0.6\linewidth]{zero_mode_sch.pdf}}
\caption{zero mode with capacitor}
\label{fig_zero_mode_sch}
\end{figure}

Figure 4 shows a proposed circuit design that the photodiode is working in zero-mode.  Experiments are done to measure the parameters of the photodiode in zero-mode, such as the saturation current, the ideality factor, the slope indicator, and the power to current conversion efficiency.  With these parameters, a small AC signal equivalent circuit can be derived.  The transfer function of this circuit indicates that it works as a bandpass filter, with the lower cut-off frequency depending on the photodiode capacitance Cj and capacitor C.  Large capacitor C is used so that the photodiode capacitance Cj can be ignored, and the filter characteristics can be calculated.  All the DC parts in the light will be filtered out by this bandpass filter as intended.
The zero-mode design could have better noise performance than the conventional design.  For both designs to have the same output AC gain, the conventional design would need a second stage amplifier, while the zero-mode design only amplifies the AC part in the first stage.  Therefore, zero-mode design can have one less amplifier stage noise.  Even when comparing only the first stage, the zero-mode design still results in less noise.  The reverse dark current and power supply interference are caused by biasing the photodiode, which does not happen on zero-mode design.  Experiments are conducted and better noise performance for zero-mode is proved.
In the experiment, the light circuit and data acquisition instrument are kept unchanged, while two circuit designs are tested under the same conditions.  With only one stage in both circuits, the photoconductive-mode design can achieve a maximum gain of 5.55 with 613uV input referred noise, while the zero-mode design achieves a maximum gain of 209.9 with only 24.3uV input referred noise.
In conclusion, this zero-mode design can achieve better noise performance compared to the conventional photoconductive-mode design.  It can also achieve a higher gain in the first stage.  Therefore, from both noise and economic point view, it is a very good design choice for the light receiver circuit.

\section{Amplifier}

The existing optrode system has a light receiver board that works under a maximum gain of \qty{120}{dB}.  For the new light receiver board that has the photodiodes working in zero-mode, we need higher gain in the amplifiers because the photodiodes has less sensitivity under zero-mode.  Also, we will need to minimize the noise in all the components in the circuit after the optrode.  Therefore, we need a high-gain low-noise transimpedance amplifier.

While there is transimpedance amplifier \cite{TIA_1} designed decades ago that can achieve \qty{3}{pA/\sqrt{Hz}}, and more recent design \cite{TIA_2} \cite{TIA_3} that can achieve as low as \qty{1}{fA/\sqrt{Hz}}, we do not have the time or resources to design and fabricate a amplifier chip by ourselves.


\section{PCB design}

Mixed-signal Printed Circuit Boards (PCBs) combine both analog and digital components, leading to the necessity of managing the interactions between these parts effectively. A prevailing challenge in this design space is electromagnetic interference (EMI), notably arising from high-frequency digital signals that might contaminate analog pathways.  There are a few techniques to apply on the PCB to minimize these effects \cite{MixPCB} \cite{MixPCB_2} \cite{MixPCB_3}.

One foundational principle in addressing EMI is recognizing that every signal requires a return path on its ground plane. It's suggested that creating distinct ground areas for both analog and digital sections can deter interference. Further, the concept of a bridge or connection between these separate areas aids in managing return currents and signal flow.

In multi-layer PCB designs, the complexity rises. It becomes vital to separate signal and power layers with ground planes. An interesting technique proposed by some researchers is the orthogonal routing of traces on adjacent signal layers to decrease interference.

Another area prone to interference is where power sections overlap with ground sections. Avoiding such overlaps is essential as they can lead to unwanted radio frequency (RF) emissions and noise. In addition, placing decoupling capacitors near power pins within each power area, whether analog or digital, has been advocated.

Physical placement considerations also extend to Digital-to-Analog Converters (DACs) and Analog-to-Digital Converters (ADCs). Positioning these components at the intersection of analog and digital areas, especially around the connection or bridge, appears to be beneficial. Moreover, when using a primary clock signal for multiple analog sections, routing the clock trace through different bridges and connections can minimize interference.

In conclusion, while EMI in mixed-signal PCBs is a challenging issue, understanding and implementing various strategies can help designers minimize potential interference, especially in designs where achieving low noise is of paramount importance.


\section{Low noise source for light source}

A superluminescent diode is used as light source for the optrode system, which has generated most of the noise in the system.  Although there is noise generated by the superluminescent diode itself \cite{SLDNoise}, a lot of noise comes from the power source of the superluminescent diode.  An alternative way to reduce the noise from this system is to have a lower noise current source.  Grzegorz \cite{LowNoiseCurrentSource} presents a low noise two-stage current source designed for laser usage, which can be similarly used on superluminescent diode.  This design creates very low noise output current by using a normal first stage source and feedback second stage source.  The result shows that at some output levels, the noise inside the output current is low enough that it will not have a significant impact on the laser \cite{LinewidthQuantumCascadeLaser}.  Therefore, this design is suitable to power some laser diodes for high performance measurement systems.

This low noise current source design has two stages.  The first stage is a normal current source mainly made by a linear voltage regulator, and the output current can be adjusted by a trimmer.  The second stage is a negative feedback loop that provides much less amount of current than the first stage, but enough to compensate for the noise from the first stage.  The second stage consists of an amplifier, a voltage reference, and a comparator.  Both the current source outputs are connected to the anode of the laser diode, and a sense resistor is placed between the cathode of the laser diode and ground, the amplifier amplifies the voltage across the resistor, which is proportional to the laser diode current.  The amplifier output voltage is compared with a highly stable voltage reference, which is adjusted as to the desired output current.  When the amplifier output voltage is smaller than the voltage reference, that means the output current is smaller than the target current because of the noise from the first stage current source, the comparator output will turn on a MOSFET to increase the second stage current, which will compensate for the noise.  Therefore, a stable low noise current is created.

Comparing the current noise for the first stage source only and the two-stage source, at lower output of I = 200mA, the two-stage source has a noise level improvement by a factor of 10 across the frequency range from 0.1Hz to 100kHz.  However, at maximum output I = 1A, the two-stage source only improves in the frequency range from 0.1Hz to 100Hz, and it has more noise than the first stage in the frequency range from 100Hz to 100kHz.
For the current noise measurement result, the two-stage source has equivalent current noise of 1.2nA/√Hz at I = 0.2A, 7.3nA/√Hz at I = 0.5A, 60nA/√Hz at I = 1A.

At a lower output current of 0.2A, the 1.2nA/√Hz noise from the current source causes lower noise in the laser diode than the temperature caused noise \cite{LinewidthQuantumCascadeLaser}.  Therefore, there is no need to improve the performance at this output noise level.  However, at higher current output current levels, the noise level increases significantly.  From the spectral density result, higher current output increases the noise in the high frequency range, this might be caused by the feedback loop gain control and reaction speed, which cause the two-stage source to be less stable at higher frequencies.  One solution to this problem is to use a feedforward loop instead of a feedback loop.  By accurately sensing and calculating the noise from the first stage source, outputting this amount of current noise from the second stage source, adding both stage current and then outputting to the laser diode.  The difference between feedback and feedforward loop is that in feedforward loop, the current noise is compensated before the laser diode, and it is accurate feed, which is more stable at high frequency than the negative feedback.

This current source is designed to have an adjustable output current.  However, the adjustment is made by tuning two trimmers according to two equations from the first and second stage sources.  As trimmers have no resistance mark, it is very hard to adjust them accurately.  Therefore, a microcontroller could be introduced to control the variable output current \cite{Du_2018}.  By using a microcontroller, both the speed and accuracy of setting the current output can be improved.  In the case of the feedforward loop, a microcontroller will be needed to calibrate the feedforward circuit gain to achieve better noise performance.

The sense resistor in this design is placed between the cathode of the laser diode and the ground.  This may cause the voltage across the laser diode to change along with the noisy current, which causes noise and drift for the laser diode.  This problem happens because of the feedback loop.  By changing from a feedback loop to feedforward, the current is sensed before the laser diode, which means the sense resistor will be placed between the first stage source and the anode of the laser diode, and the cathode of the laser diode will be grounded.  The ground connection issue for the amplifier can be solved by a high side sensing circuit \cite{TI_2019}.
 
In conclusion, Grzegorz presents a low noise current source design that achieves decent results of 1.2nA/√Hz noise at I = 0.2A output, which is approaching the limit of the noise performance improvement ability of the laser diode from the current source.  However, the current source noise performance at higher output currents is less satisfying.  The feedback loop has a performance impact on higher frequency noise.  Also, the output current value takes a long time to change.  His work informs my research in which a feedforward loop second stage current source will be implemented, and the whole source will be controlled by a microcontroller.



























 












